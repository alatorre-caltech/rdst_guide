\documentclass[12pt]{report}
\title{R(D*) Analysis Guide}
\author{Anthony LaTorre}
\usepackage{hyperref}
\usepackage{caption}
\usepackage{listings}
\lstset{
  basicstyle=\ttfamily,
  columns=fullflexible,
  frame=none,
  breaklines=true,
  postbreak=\mbox{\textcolor{red}{$\hookrightarrow$}\space},
}
\usepackage{xcolor}
\usepackage[framemethod=TikZ]{mdframed}
\usepackage{fullpage}
\definecolor{light-gray}{gray}{0.95} %the shade of grey that stack exchange uses
\begin{document}
\maketitle
\tableofcontents
\begin{abstract}
This is a short document intended to provide an intro to the R(D*) analysis and talk about some of the current problems.
\end{abstract}
\section{Software}
The R(D*) analysis software consists of three repositories:
\begin{enumerate}
\item \href{https://github.com/alatorre-caltech/BPH\_RD\_Analysis}{BPH\_RD\_Analysis}
\item \href{https://github.com/alatorre-caltech/BPH\_RDntuplizer}{BPH\_RDntuplizer}
\item \href{https://github.com/alatorre-caltech/BPH\_CMSMCGen}{BPH\_CMSMCGen}
\end{enumerate}
\subsection{Getting Started}
Before running any of the software it is necessary to set up your environment.
These instructions will assume you are running things on the Caltech Tier2
login nodes, but the instructions should be similar for any computer.

First, we are going to set up a container that will run RHEL7 since that is
what the software was developed on\footnote{I'm not actually sure if there is
any good reason to have everything running under RHEL7. It makes a lot of
things much more painful, and so if everything can compile and give identical
results under Alma Linux 8 that would be much much better.}.

\begin{mdframed}[backgroundcolor=light-gray, roundcorner=10pt,leftmargin=1, rightmargin=1, innerleftmargin=15, innertopmargin=15,innerbottommargin=15, outerlinewidth=1, linecolor=light-gray,roundcorner=20pt]
\begin{lstlisting}
$ mkdir ~/bin
$ echo "export PATH=$HOME/bin:$PATH" >> ~/.bash_profile
$ echo "source /cvmfs/cms.cern.ch/cmsset_default.sh" >> ~/.bash_profile
$ source ~/.bash_profile
$ cp `which cmssw-cc7` ~/bin
\end{lstlisting}
\end{mdframed}

Then, edit the file \texttt{~/bin/cmssw-cc7} and add the following line:

\begin{mdframed}[backgroundcolor=light-gray, roundcorner=10pt,leftmargin=1, rightmargin=1, innerleftmargin=15, innertopmargin=15,innerbottommargin=15, outerlinewidth=1, linecolor=light-gray,roundcorner=20pt]
\begin{lstlisting}
SINGULARITY_BINDPATH=$SINGULARITY_BINDPATH,/storage/:/storage/
\end{lstlisting}
\end{mdframed}

just before the last line, i.e.

\begin{mdframed}[backgroundcolor=light-gray, roundcorner=10pt,leftmargin=1, rightmargin=1, innerleftmargin=15, innertopmargin=15,innerbottommargin=15, outerlinewidth=1, linecolor=light-gray,roundcorner=20pt]
\begin{lstlisting}
[...]
SINGULARITY_BINDPATH=$SINGULARITY_BINDPATH,/storage/:/storage/
singularity -s exec ${SINGULARITY_OPTS} $UNPACKED_IMAGE sh -c "${CMD_TO_RUN[@]}"
\end{lstlisting}
\end{mdframed}

Now we can create the main directory that everything will be put under, and
create the CMSSW environments needed:
\begin{mdframed}[backgroundcolor=light-gray, roundcorner=10pt,leftmargin=1, rightmargin=1, innerleftmargin=15, innertopmargin=15,innerbottommargin=15, outerlinewidth=1, linecolor=light-gray,roundcorner=20pt]
\begin{lstlisting}
$ cd
$ cmssw-cc7
Singularity> source /cvmfs/cms.cern.ch/cmsset_default.sh
Singularity> mkdir RDstAnalysis
Singularity> cd RDstAnalysis
Singularity> export SCRAM_ARCH=slc7_amd64_gcc700
Singularity> cmsrel CMSSW_10_2_3  # For the ntuplizer
Singularity> cmsrel CMSSW_10_2_13 # For combine
Singularity> # Now, we get out of the container and set up a final environment for when we
Singularity> # don't need to run anything platform specific
Singularity> exit
$ cd ~/RDstAnalysis
$ export SCRAM_ARCH=cc8_amd64_gcc9
$ cmsrel CMSSW_11_2_0
\end{lstlisting}
\end{mdframed}
It is important to use the exact same directory structure since this is assumed
in much of the code.

First, we are going to add some basic stuff to our .bashrc so that we have
access to the CMS programs, etc. every time we log in. To do so add the
following to your .bashrc when you log in:

\begin{mdframed}[backgroundcolor=light-gray, roundcorner=10pt,leftmargin=1, rightmargin=1, innerleftmargin=15, innertopmargin=15,innerbottommargin=15, outerlinewidth=1, linecolor=light-gray,roundcorner=20pt]
\begin{lstlisting}
ulimit -s unlimited

PATH=$HOME/bin:$PATH:$HOME/.local/bin

export PATH
export EDITOR=vim

export HISTFILESIZE=
export HISTSIZE=

if [[ $- == *i* ]]; then
    source /cvmfs/cms.cern.ch/cmsset_default.sh
    export X509_USER_CERT=$HOME/.globus/usercert.pem
    export X509_USER_KEY=$HOME/.globus/private/userkey.pem
    export X509_USER_PROXY=/tmp/x509up_u${EUID}
    export PATH=/cvmfs/sft.cern.ch/lcg/contrib/CMake/3.14.2/Linux-x86_64/bin:${PATH}
    export BOOST_ROOT=/cvmfs/sft.cern.ch/lcg/releases/Boost/1.66.0-f50b5/x86_64-centos7-gcc7-opt/
    export HEPMC_DIR=/cvmfs/sft.cern.ch/lcg/external/HepMC/2.06.08/x86_64-slc6-gcc48-opt
    cd $HOME/RDstAnalysis/CMSSW_11_2_0/
    eval `scramv1 runtime -sh`
    source $HOME/RDstAnalysis/BPH_RD_Analysis/env.sh
    cd -
fi

\end{lstlisting}
\end{mdframed}

\subsubsection{Installing the ntuplizer}

\subsection{BPH\_RD\_Analysis}
The BPH\_RD\_Analysis contains the following scripts:
\begin{itemize}
\item B2DstMu\_skimCAND\_v1.py which converts the ntuples for the normal B -> D* mu nu analysis into ``skimmed'' data files which are used as input to the final fit
\item B2JpsiKst\_skimCAND\_v1.py which converts the ntuples for the calibration samples into ``skimmed'' data files
\item generatorEfficiency.py which looks at the MiniAOD logs to produce text files which contain the efficiency of tagging a given sample. These efficiencies are used in the final fit when computing the overall normalization for each sample.
\item triggerEfficiencies.py and triggerEfficienciesScaleFactors.py are used to compute the trigger efficiency corrections.
\item kinematicCalibration\_Bd\_JpsiKst.py is used to compute the corrections for the B pT and extra track pT, etc. from the calibration sample
\item forcedDecayChannelsFactors\_v2.ipynb is used to calculate the normalization of the branching ratios for each sample based on what decays were forced in the MC card
\end{itemize}
\subsubsection{Running the Skimmer}
To produce ``skimmed'' data files from the ntuples, you can run:
\begin{mdframed}[backgroundcolor=light-gray, roundcorner=10pt,leftmargin=1, rightmargin=1, innerleftmargin=15, innertopmargin=15,innerbottommargin=15, outerlinewidth=1, linecolor=light-gray,roundcorner=20pt]
\begin{lstlisting}
$ python B2DstMu_skimCAND_v1.py -d '.*' --cat low
$ python B2DstMu_skimCAND_v1.py -d '.*' --cat mid
$ python B2DstMu_skimCAND_v1.py -d '.*' --cat high
\end{lstlisting}
\end{mdframed}
Note that it's possible to run these three in parallel. To run the same for the calibration data:
\begin{mdframed}[backgroundcolor=light-gray, roundcorner=10pt,leftmargin=1, rightmargin=1, innerleftmargin=15, innertopmargin=15,innerbottommargin=15, outerlinewidth=1, linecolor=light-gray,roundcorner=20pt]
\begin{lstlisting}
$ python B2JpsiKst_skimCAND_v1.py -d '.*' --cat low
$ python B2JpsiKst_skimCAND_v1.py -d '.*' --cat mid
$ python B2JpsiKst_skimCAND_v1.py -d '.*' --cat high
\end{lstlisting}
\end{mdframed}
\subsubsection{Computing the Trigger Efficiency}
To compute the trigger efficiencies for data and MC you can run:
\begin{mdframed}[backgroundcolor=light-gray, roundcorner=10pt,leftmargin=1, rightmargin=1, innerleftmargin=15, innertopmargin=15,innerbottommargin=15, outerlinewidth=1, linecolor=light-gray,roundcorner=20pt]
\begin{lstlisting}
$ for t in "Mu7_IP4" "Mu9_IP6" "Mu12_IP6"; do ./triggerEfficiencies.py -v [version] -t $t -d RD --refIP BS; done
$ for t in "Mu7_IP4" "Mu9_IP6" "Mu12_IP6"; do ./triggerEfficiencies.py -v [version] -t $t -d MC --refIP BS; done
$ for t in "Mu7_IP4" "Mu9_IP6" "Mu12_IP6"; do ./triggerEfficienciesScaleFactors.py -v [version] -t $t --refIP BS; done
\end{lstlisting}
\end{mdframed}
\subsubsection{Editing IPython Notebooks}
To edit ipython notebooks from a remote server like login-2, it is necessary to do some port forwarding. Here is how I do it (for example: to modify forcedDecayChannelsFactors\_v2.ipynb).
\begin{mdframed}[backgroundcolor=light-gray, roundcorner=10pt,leftmargin=1, rightmargin=1, innerleftmargin=15, innertopmargin=15,innerbottommargin=15, outerlinewidth=1, linecolor=light-gray,roundcorner=20pt]
\begin{lstlisting}
$ ssh login-2.hep.caltech.edu
$ cd RDstAnalysis/BPH_RD_Analysis/scripts
$ jupyter-notebook --no-browser --port 1234
\end{lstlisting}
\end{mdframed}
Then, on your local machine:
\begin{mdframed}[backgroundcolor=light-gray, roundcorner=10pt,leftmargin=1, rightmargin=1, innerleftmargin=15, innertopmargin=15,innerbottommargin=15, outerlinewidth=1, linecolor=light-gray,roundcorner=20pt]
\begin{lstlisting}
$ ssh ssh -NL 1234:localhost:1234 login-2.hep.caltech.edu
\end{lstlisting}
\end{mdframed}
Then you should be able to visit the link that the jupyter notebook gives you
on your local machine.
\subsection{BPH\_RDntuplizer}
\end{document}
