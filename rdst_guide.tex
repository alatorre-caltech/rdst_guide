\documentclass[12pt]{article}
\title{R(D*) Analysis Guide}
\author{Anthony LaTorre}
\usepackage{hyperref}
\usepackage{caption}
\usepackage{listings}
\lstset{
  basicstyle=\ttfamily,
  columns=fullflexible,
  frame=none,
  breaklines=true,
  postbreak=\mbox{\textcolor{red}{$\hookrightarrow$}\space},
}
\usepackage{xcolor}
\usepackage[framemethod=TikZ]{mdframed}
\usepackage{fullpage}
\definecolor{light-gray}{gray}{0.95} %the shade of grey that stack exchange uses
\begin{document}
\maketitle
\begin{abstract}
This is a short document intended to provide an intro to the R(D*) analysis and talk about some of the current problems.
\end{abstract}
\section{Software}
The R(D*) analysis software consists of three repositories:
\begin{enumerate}
\item \href{https://github.com/alatorre-caltech/BPH\_RD\_Analysis}{BPH\_RD\_Analysis}
\item \href{https://github.com/alatorre-caltech/BPH\_RDntuplizer}{BPH\_RDntuplizer}
\item \href{https://github.com/alatorre-caltech/BPH\_CMSMCGen}{BPH\_CMSMCGen}
\end{enumerate}
\subsection{BPH\_RD\_Analysis}
The BPH\_RD\_Analysis contains the following scripts:
\begin{itemize}
\item B2DstMu\_skimCAND\_v1.py which converts the ntuples for the normal B -> D* mu nu analysis into ``skimmed'' data files which are used as input to the final fit
\item B2JpsiKst\_skimCAND\_v1.py which converts the ntuples for the calibration samples into ``skimmed'' data files
\item generatorEfficiency.py which looks at the MiniAOD logs to produce text files which contain the efficiency of tagging a given sample. These efficiencies are used in the final fit when computing the overall normalization for each sample.
\item triggerEfficiencies.py and triggerEfficienciesScaleFactors.py are used to compute the trigger efficiency corrections.
\item kinematicCalibration\_Bd\_JpsiKst.py is used to compute the corrections for the B pT and extra track pT, etc. from the calibration sample
\item forcedDecayChannelsFactors\_v2.ipynb is used to calculate the normalization of the branching ratios for each sample based on what decays were forced in the MC card
\end{itemize}
\subsubsection{Running the Skimmer}
To produce ``skimmed'' data files from the ntuples, you can run:
\begin{mdframed}[backgroundcolor=light-gray, roundcorner=10pt,leftmargin=1, rightmargin=1, innerleftmargin=15, innertopmargin=15,innerbottommargin=15, outerlinewidth=1, linecolor=light-gray,roundcorner=20pt]
\begin{lstlisting}
$ python B2DstMu_skimCAND_v1.py -d '.*' --cat low
$ python B2DstMu_skimCAND_v1.py -d '.*' --cat mid
$ python B2DstMu_skimCAND_v1.py -d '.*' --cat high
\end{lstlisting}
\end{mdframed}
Note that it's possible to run these three in parallel. To run the same for the calibration data:
\begin{mdframed}[backgroundcolor=light-gray, roundcorner=10pt,leftmargin=1, rightmargin=1, innerleftmargin=15, innertopmargin=15,innerbottommargin=15, outerlinewidth=1, linecolor=light-gray,roundcorner=20pt]
\begin{lstlisting}
$ python B2JpsiKst_skimCAND_v1.py -d '.*' --cat low
$ python B2JpsiKst_skimCAND_v1.py -d '.*' --cat mid
$ python B2JpsiKst_skimCAND_v1.py -d '.*' --cat high
\end{lstlisting}
\end{mdframed}
\subsubsection{Computing the Trigger Efficiency}
To compute the trigger efficiencies for data and MC you can run:
\begin{mdframed}[backgroundcolor=light-gray, roundcorner=10pt,leftmargin=1, rightmargin=1, innerleftmargin=15, innertopmargin=15,innerbottommargin=15, outerlinewidth=1, linecolor=light-gray,roundcorner=20pt]
\begin{lstlisting}
$ for t in "Mu7_IP4" "Mu9_IP6" "Mu12_IP6"; do ./triggerEfficiencies.py -v [version] -t $t -d RD --refIP BS; done
$ for t in "Mu7_IP4" "Mu9_IP6" "Mu12_IP6"; do ./triggerEfficiencies.py -v [version] -t $t -d MC --refIP BS; done
$ for t in "Mu7_IP4" "Mu9_IP6" "Mu12_IP6"; do ./triggerEfficienciesScaleFactors.py -v [version] -t $t --refIP BS; done
\end{lstlisting}
\end{mdframed}
\subsubsection{Editing IPython Notebooks}
To edit ipython notebooks from a remote server like login-2, it is necessary to do some port forwarding. Here is how I do it (for example: to modify forcedDecayChannelsFactors\_v2.ipynb).
\begin{mdframed}[backgroundcolor=light-gray, roundcorner=10pt,leftmargin=1, rightmargin=1, innerleftmargin=15, innertopmargin=15,innerbottommargin=15, outerlinewidth=1, linecolor=light-gray,roundcorner=20pt]
\begin{lstlisting}
$ ssh login-2.hep.caltech.edu
$ cd RDstAnalysis/BPH_RD_Analysis/scripts
$ jupyter-notebook --no-browser --port 1234
\end{lstlisting}
\end{mdframed}
Then, on your local machine:
\begin{mdframed}[backgroundcolor=light-gray, roundcorner=10pt,leftmargin=1, rightmargin=1, innerleftmargin=15, innertopmargin=15,innerbottommargin=15, outerlinewidth=1, linecolor=light-gray,roundcorner=20pt]
\begin{lstlisting}
$ ssh ssh -NL 1234:localhost:1234 login-2.hep.caltech.edu
\end{lstlisting}
\end{mdframed}
Then you should be able to visit the link that the jupyter notebook gives you
on your local machine.
\subsection{BPH_RDntuplizer}
\end{document}
