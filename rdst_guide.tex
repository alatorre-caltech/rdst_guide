\documentclass[12pt]{article}
\title{R(D*) Analysis Guide}
\author{Anthony LaTorre}
\begin{document}
\maketitle
\begin{abstract}
This is a short document intended to provide an intro to the R(D*) analysis and talk about some of the current problems.
\end{abstract}
\section{Introduction}
The R(D*) analysis software consists of three repositories:
\begin{enumerate}
\item \href{BPH\_RD\_Analysis}{https://github.com/alatorre-caltech/BPH_RD_Analysis}
\item \href{BPH\_RDntuplizer}{https://github.com/alatorre-caltech/BPH_RDntuplizer}
\item \href{BPH\_CMSMCGen}{https://github.com/alatorre-caltech/BPH_CMSMCGen}
\end{enumerate}
\subsection{BPH\_RD\_Analysis}
The BPH\_RD\_Analysis contains the following scripts:
\begin{itemize}
\item B2DstMu\_skimCAND\_v1.py which converts the ntuples for the normal B -> D* mu nu analysis into ``skimmed'' data files which are used as input to the final fit
\item B2JpsiKst\_skimCAND\_v1.py which converts the ntuples for the calibration samples into ``skimmed'' data files
\item generatorEfficiency.py which looks at the MiniAOD logs to produce text files which contain the efficiency of tagging a given sample. These efficiencies are used in the final fit when computing the overall normalization for each sample.
\item triggerEfficiencies.py and triggerEfficienciesScaleFactors.py are used to compute the trigger efficiency corrections.
\item kinematicCalibration_Bd_JpsiKst.py is used to compute the corrections for the B pT and extra track pT, etc. from the calibration sample
\item forcedDecayChannelsFactors_v2.ipynb is used to calculate the normalization of the branching ratios for each sample based on what decays were forced in the MC card
\end{itemize}
\end{document}
